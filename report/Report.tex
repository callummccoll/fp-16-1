% version 1

\documentclass[10point]{article}

% prelude

\usepackage{dirtree}
\usepackage{fancyvrb}
\usepackage{graphicx}
\usepackage[pdftex, backref, colorlinks, bookmarksnumbered=true]{hyperref}

\usepackage{listings}
%\lstloadlanguages{Haskell}
%\lstnewenvironment{code}
%{\lstset{}%
%    \csname lst@SetFirstLabel\endcsname}
%{\csname lst@SaveFirstLabel\endcsname}
%\lstnewenvironment{examplecode}
%{\lstset{}%
%    \csname lst@SetFirstLabel\endcsname}
%{\csname lst@SaveFirstLabel\endcsname}
%\lstset{
%        basicstyle=\ttfamily,
%        flexiblecolumns=false,
%        basewidth={0.5em,0.45em}
%}

\DefineVerbatimEnvironment{code}{Verbatim}{}
\DefineVerbatimEnvironment{examplecode}{Verbatim}{}

\newcommand{\highlight}[1]{\colorbox{yellow}{#1}}
\newcommand{\highlighttt}[1]{\highlight{{\tt#1}}}
%\newcommand{\lhs}[1]{{\lstset{
%            literate={+}{{$+$}}1 {/}{{$/$}}1 {*}{{$*$}}1 {=}{{$=$}}1
%            {>}{{$>$}}1 {<}{{$<$}}1 {\\}{{$\lambda$}}1
%            {\\\\}{{\char`\\\char`\\}}1
%            {->}{{$\rightarrow$}}2 {>=}{{$\geq$}}2 {<-}{{$\leftarrow$}}2
%            {<=}{{$\leq$}}2 {=>}{{$\Rightarrow$}}2 
%            {\ .}{{$\circ$}}2 {\ .\ }{{$\circ$}}2
%            {>>}{{>>}}2 {>>=}{{>>=}}3
%            {|}{{$\mid$}}1
%            {>|>}{{>|>}}3 {>>|>}{{>>|>}}4 {>>|>>}{{>>|>>}}5 {>|>>}{{>|>>}}4
%            {<|<}{{<|<}}3 {<<|<}{{<<|<}}4 {<<|<<}{{<<|<<}}5 {<|<<}{{<|<<}}4
%            {>:=}{{>:=}}3 {>>:=}{{>>:=}}4
%            {\\n}{{\tt{$\\\\$\\n}}}2   
%        }
%        #1
%}}

\long\def\ignore#1{}

\begin{document}

\title{Visuals Report}
\author{Callum McColl\\
	    Andrew Paroz}
		
\maketitle

\tableofcontents

\listoffigures

\section{Introduction}
The goal of this project was to create a program that could compile, execute and displaye a C program. It could then be used to go step-by-step through a C program in order to show how the memory and registers are effected by each line of C code. In order to learn functional programming and Mash as a template, this program was created using Haskell. The project could be divided into 5 separate sections, the C parser, the assembly parser, C to assembly code generation, the emulator and the GUI. These were then allocated to three groups, one group would create the C Parser, another the Assembly Parser and Symbol Table, and the last group the Emulator and GUI. The Code generation which converts a C Parse tree to an Assembly Parse tree was not assigned to any group. A diagram of this can be see in Figure~\ref{fig:ProjectDiagram}.

\begin{figure}[h]
\centering
\includegraphics[width=300px]{ProjectDiagram}
\caption{Project diagram showing all the main parts and group allocations.}
\label{fig:ProjectDiagram}
\end{figure}

The program that this project created is to be used as a learning tool to help teach C programming. C is a difficult language when starting as there are multiple ways to do the same thing in the language, but some ways are more 'right' than others. The compilers can also be very lenient and compile and run code that is not correct, leading to problems where code ends up running differently on different systems. By having a program that can go step-by-step through a C program, it will be much easier to show where errors are, and explain to students why a program is acting in that way.

Obviously the C language is very vast, so only a small subset was chosen to be implemented for this project. All variables are either an Int or a pointer, there are no structs and everything must be contained within a single file. This allowed the assembler and emulator to be much simpler and made it possible to complete in the time we had.

Our group was the Visuals Group and was assigned to create the Emulator and GUI. Our project was significantly different from the other two groups which were creating parsers. In general, Callum was responsible for creating the GUI and Andrew created the Emulator and Environment. There was overlap between those sections when designing the programming interface for the emulator and environment, however most of the work was separate. Since the emulator uses an assembly parse tree, we also had to work with the group creating that in order to connect them together.

\section{Process and Design}

\subsection{Process}
This project was approached using a prototyping methodology, where a design would be implemented, then assessed and modified or discarded. This approach was used since multiple different solutions were initially suggested for the implementation and each had to be tested and assessed. This method was also useful for filling in the blanks at the start of the project in regards to how it would work with the assembly parse tree, and how the GUI would interact with the emulator. This prototype method meant we could create a template for a part based on how it may be designed, then easily update the program once that part was actually complete.

In order to easily share and work together, a Google Drive folder was set up, and for the visuals, a separate Git repository was made.  This was essential for the visuals as the application needed to be tested on Windows, Unix and Posix and each group member only had access to a subset of these platforms.  Therefore there were many times in the project where a group member would ask the other to pull the repository and verify that the behaviour exhibited by the program was correct.

Another benefit for using the repository was that it gave everyone developing the latest changes so it was known immediately if something that the other had changed had inadvertently affected the others work.  Overall the Git repository gave each developer the freedom to work individually without the worry of having to merge two separate implementation at a later date.

\subsection{Environment}
The creation of the Environment was an important early step as it defined the connection between the Emulator and the GUI. An Environment represents a single state of an emulated system, with memory, registers, inputs, outputs and the symbol table. The GUI can then use all the information in an Environment to display the state of a machine, while the Emulator can use that same information to emulate the machine running. Its data structures were mostly designed during an in-class meeting, then it was extended with many utility functions in order to increase its usefulness. 

Since the Environment was created at the start of the project, it was possible to generate a dummy environment with handwritten values in order to test the functionality of the Emulator and GUI. Once the assembly parse tree was completed however, this was replaced with a function that created an environment based on a parse tree and symbol table. This could be used to create the initial state of a system from an assembly file.

Before the parse tree was integrated, the assembly instructions were stored as plain strings, which made it easy to display in the GUI. However with the parse tree, the instructions were now data classes, so six utility functions were created that could give back an instruction as a string. This allowed the functions to be displayed in the way it was expected in the GUI and within error messages in the Emulator.

The last change to the Environment came at the end of the project when it was found we needed a flag to halt emulation and let the GUI know there were no more states. Originally the PC was used to figure out if the emulation was finished, if the PC did not change between two states, it must meant the program was finished. This was changed to a flag in the environment that was set to true when either the HALT command was run, or the program tried to read input and there was none.

\subsection{Emulator}
The emulator took a significant amount of time over the project and involved reading and executing assembly instructions to a virtual environment. Because this was a rather complex and large piece of code, it was broken in parts. First an instruction was read from memory based on the PC, the type of instruction was then used to call an associated function, for example, a MOVE instruction could call the actionMove function. Each function would then do what it needed to based on he parameters it was given. The largest part of the emulator was the destination and source functions, which needed to handle a wide range of different types of sources and destinations, each with unique interactions.

The first emulator was created without knowing what the assembly parse tree looked like, and was based off instructions given as strings. While this made the pattern matching for instructions easy, it lead to a huge list of permutations for both the source and destinations. Once the parse tree was available, all the strings were replaced with data types from the parse tree, which made calling the source and destination functions a little trickier, but improved their ability to be recursive with themselves.

Before the environment was designed, the emulator stored memory as just a list of strings. With the environment however, the memory became an IOArray, which made reading and writing to it much easier and it was then possible to error check during emulation. If the program tried to read an instruction from a memory location where an Int or nothing was stored, a descriptive error could be printed. This was obviously very useful at the end of the project to make sure the emulator was working correctly with various test programs. The only bugs found when testing with the GUI revolved around simple logic mistakes with multiple levels of indirection. 

Upon review at the end of the project, many 'actions' within the emulator are repetitive, such as the maths operations and the decision functions. With a little work, these repetitive functions could be combined into a single one. This was a leftover from when string pattern matching was being used to read instructions and these functions were more different from each other.

\subsection{Graphical User Interface (GUI)}
\subsubsection{Cross-Platform GUI Solution}
The first task of our group was to determine how we would create a GUI using Haskell that could be completely cross-platform. Two methods were initially recommended, using Haskell to generate a HTML page that displayed everything, or using the GTK3 Hakell library to create an application window.

The first test was using a HTML page with a small amount of JavaScript to make it interactive. A screenshot of the example page can be seen in figure~\ref{fig:WebTest}. The idea was that once a program was compiled, it could generate this page with the all the program states loaded in as data. By using HTML, this file could then be opened on any system and the program could be looked at again without needing to be recompiled. However, it also had the huge drawback that the HTML page could not communicate with the emulator, which was too big a drawback for this project.

\begin{figure}[h!]
\centering
\includegraphics[width=300px]{WebScreenshot}
\caption{Screenshot of Web display.}
\label{fig:WebTest}
\end{figure}

The next step was to test GTK3, and an example application was also made for it. This test program used Glade to structure the application window and GTK3 to draw it to the screen. A screenshot of the test program can be seen in figure~\ref{fig:GTK3Test}. As a demo, it had very limited interactivity, allowing you to  step through a handwritten list of program states and nothing could be modified. It did demonstrate that GTK3 would indeed be suitable for the project, however the use of Glade to position elements on the window was not very useful as most of the windows ended up being built using code.

\begin{figure}[h!]
\centering
\includegraphics[width=300px]{GTK3TestScreenshot}
\caption{Screenshot of GTK3 test program.}
\label{fig:GTK3Test}
\end{figure}

\newpage

\subsubsection{The Final Product}
While using Glade supplied a very nice proof-of-concept implementation, it became quite difficult to manage the dynamic nature of the program.  Therefore it was decided that attempting to complete the project using Glade was dropped for an entirely code based approach.

The final product follows from the proof-of-concept implementation but does of course emulate the assembly source code that it is given.  Unfortunately the C parsing was not completed so it was decided to abandon the C frame which is shown in the proof-of-concept implementation for more screen real estate.  The GUI has a main menu bar at the top which can be used to open other assembly source files.  Below the menu is a tool bar which presents play, stop, previous and next buttons.  This allows us to modify the assembly code and then run it much like other IDE's allow you to do.

The GUI runs in two states.  When the GUI first launches it is in "edit" mode.  This means that the assembly source code and stdin are editable.  Once the user wants to run the emulation they then press the "play" button.  This runs the emulation.  The user is then able to step through the program.

\section{Design Details}

\begin{minipage}{\linewidth}
\subsection{Directory Structure}

It is important to setup your directory structure to mimic the structure that was used during development:\newline
\dirtree{%
    .1 machine.
    .2 parser.
    .3 ABR.
    .3 assembler.hs.
    .3 Assembly.hs.
    .3 SymbolTable.hs.
    .1 Visuals.
    .2 Containers.lhs.
    .2 countdown.ass.
    .2 Emulation.lhs.
    .2 Environment.lhs.
    .2 Helpers.lhs.
    .2 Icons.lhs.
    .2 Main.lhs.
    .2 Makefile.
    .2 Menus.lhs.
    .2 Operators.lhs.
    .2 Presentation.lhs.
    .2 Ram.lhs.
    .2 README.md.
    .2 test.ass.
}
\end{minipage}

\noindent \newline If you don't wish to follow this directory structure then you should change where the Makefile is looking for the assembly parser:

\begin{examplecode}
HCFLAGS = -fno-warn-tabs -ipath/to/machine/parser -XPackageImports
\end{examplecode}

\subsection{Compiling on Windows}
Although GTK3 is a cross platform library, getting everything installed on windows to compile it with Haskell is actually quite hard. This tutorial was written based off instructions from multiple websites that were all out of date. 
\linebreak \\
\textbf{Step 1}: Haskell on Windows
I used the Haskell Platform to install haskell on Windows. This provides haskell commands (ghc, cabal etc) through the windows command prompt. (\href{https://www.haskell.org/platform/}{https://www.haskell.org/platform/})

\noindent Though any Haskell installation for windows should work just fine for this.
\linebreak \\
\textbf{Step 2}: Get the GTK3 and Glade Binaries and dependencies
In order to get the up to date gtk3 binaries, and all the dependencies, we need to use msys. Download msys (\href{https://sourceforge.net/projects/msys2/}{https://sourceforge.net/projects/msys2/}) and install to a path without spaces. I installed to C:\\dev\\msys . You could do this without msys, but we need quite a few libraries, and they are not easy to find by themselves and it is not recommended to install them without msys.

Once installed, run msys and you'll get a console, this is what we get all the packages with. First we need to update msys. Type:
Run pacman -Syuu

Then quit the console and open it again.

Now we get the packages. Each package has a 32-bit and 64-bit version. Install the one suitable for you. I was on a 64-bit computer, so I used the 64-bit packages.

\noindent \textbf{pkg-config:}\\
pacman -S mingw-w64-i686-pkg-config    (32-bit)\\
pacman -S mingw-w64-x86\_64-pkg-config  (64-bit)\\

\noindent \textbf{cairo:}\\
pacman -S mingw-w64-i686-cairomm    (32-bit)\\
pacman -S mingw-w64-x86\_64-cairomm  (64-bit)\\

\noindent \textbf{glib:}\\
pacman -S mingw-w64-i686-glib2    (32-bit)\\
pacman -S mingw-w64-x86\_64-glib2  (64-bit)\\

\noindent \textbf{gtk2:}\\
pacman -S mingw-w64-i686-gtk2    (32-bit)\\
pacman -S mingw-w64-x86\_64-gtk2  (64-bit)\\

\noindent \textbf{gtk3:}\\
pacman -S mingw-w64-i686-gtk3    (32-bit)\\
pacman -S mingw-w64-x86\_64-gtk3  (64-bit)\\
\linebreak
\textbf{Step 3}: Haskell packages.
Now that we have all the c libraries, we can install the connected Haskell packages. First we need to link the c libraries so that Haskell can see them. To do this we need to edit a system environment variable.

Go to \verb"Computer -> System Properties ->" \\
\indent \verb"Advanced System Properties -> Environment Variables".

\noindent In the top box you'll see a variable called PATH, edit this. Values are separated by ; so add a ; to the end and then put the filepath to \verb"msys64\mingw64\bin". In my case this was \verb"C:\dev\msys64\mingw64\bin". Now open up a windows command prompt (with administer privileges).

\noindent Now for some Haskell libraries we need install. As always, use:\\
cabal update\\

\noindent Then:\\
cabal install alex\\
cabal install happy\\
cabal install gtk2hs-buildtools\\

\noindent Now it gets harder, there was/is a bug on Windows for a few of the Haskell library. glib, gio, pango and gtk3 all will fail to install directly with cabal. Try to install it first, but if you get an error about \verb"__debugbreak", then you'll have to do some editing.

Navigate to a location where we can unpack each package. I just used \verb"c:\dev\" where I installed msys and the Haskell platform, but the location doesn't matter. Then use cabal unpack to download and unpack the following packages.\\

\noindent cabal unpack glib\\
cabal unpack gio\\
cabal unpack pango\\
cabal unpack gtk\\
cabal unpack gtk3\\

\noindent Now go into each unpacked package and open the .cabal file, remove the \verb"-D__attribute__(A)=" attribute from the cpp-options.
\\ \linebreak

\noindent At the time of writing, there is a bug in gio that will cause this error: \\
\verb"System\GIO\File\IOError.chs:49:15: parse error on input '*'"
\linebreak

\noindent This bug has already been fixed in gtk2hs, but not in gio. You'll need to edit one of the source files. Find \verb"\gio-0.13.1.1\System\GIO\File\IOError.chs" and delete the \verb"{-# LANGUAGE CPP #-}" line. This file does not require cpp, and cpp ends up processing part of a comment, causing the error.
\\ \linebreak

\noindent Now manually install each of these pages, go into each unpacked package with the console and type: \\
cabal install \\
In the same order as when you unpacked them above.
\\ \linebreak

\noindent \textbf{Step 4}: Done \\
You're finally done and should be able to compile this project.

\subsection{Compiling on Mac OSX}

To setup gtk on the Mac OSX platform it is best to use the homebrew package manager:

\begin{examplecode}
    brew install gtk+
    brew install gtk+3
    brew install librsvg    
\end{examplecode}

\noindent Once gtk is installed on the system you can then use the cabal package manager to install gtk2hs and its dependencies:

\begin{examplecode}
    cabal install alex
    cabal install happy
    cabal install gtk -fhave-quartz-gtk
    cabal install gtk3
\end{examplecode}

\section{Implementation}
\subsection{Environment}
\input{../Environment.lhs}

\subsection{Emulation}
\input{../Emulation.lhs}

\subsection{GUI}

\subsubsection{GTK3}

GTK3 runs everything through a main loop.  Therefore it uses an event-driven paradigm where the program mostly does nothing until certain events are triggered.  The "main loop" is responsible for waiting for user input and invoking the appropriate listener functions.  It is very important to note that the main loop is blocking while it waits for user input.  Therefore you must do everything based on an event.

It is important to understand how GTK3 organizes its objects so we are going to talk a little about how GTK works.  GTK3 is a c library but the gtk2hs project has organized everything into a type class hierarchy.  As such the \highlighttt{GObjectClass} is the type class that every glib object conforms to.  The next one down is the \highlighttt{WidgetClass}.  Everything that can be placed on the stage conforms to \highlighttt{WidgetClass}.

If an object conforms to the \highlighttt{ContainerClass} then they are capable of containing other widgets.  \highlighttt{VBox} is a container that organizes its children into a vertical layout.  Where the first child is placed vertically above the second.  \highlighttt{HBox} is similar but it instead organizes its children in a horizontal layout.  The final object that is worth mentioning is \highlighttt{Frame}.  \highlighttt{Frame} is a container that displays a rounded cornered box with a title.

For more information on GTK I would recommend \href{http://hackage.haskell.org/package/gtk-0.12.3/docs/Graphics-UI-Gtk.html}{the API for the gtk2hs module.}

\subsubsection{Structure}

The GUI implementation is separated into several modules, Figure \ref{fig:guimodules} shows which of these modules depend on each other.

\begin{figure}[h!]
\centering
\includegraphics[width=300px]{modules}
\caption{GUI Modules}
\label{fig:guimodules}
\end{figure}


\subsubsection{Operators}
\input{../Operators.lhs}
    
\subsubsection{Helpers}
\input{../Helpers.lhs}

\subsubsection{Presentation}
\input{../Presentation.lhs}

\subsubsection{Containers}
\input{../Containers.lhs}

\subsubsection{Menus}
\input{../Menus.lhs}
    
\subsubsection{Icons}
\input{../Icons.lhs}

\subsubsection{Ram}
\input{../Ram.lhs}

\subsubsection{Main}
\input{../Main.lhs}

\section{Results, evidence of what works}
Screenshots of the GUI and output of printing an environment I guess.

\subsection{Emulation}
It is a little hard to show the end result of running an emulation, however figure~\ref{fig:EmulationResult} shows a printout of the final state of a program once it has been emulated. Every instruction that can be parsed has been tested to make sure it works in the way expected from the original machine design that was created.

\begin{figure}[h]
    \centering
    \includegraphics[width=300px]{EmulationResult}
    \caption{Final state of countdown.ass after emulation.}
    \label{fig:EmulationResult}
\end{figure}

The assembly program "countdown.ass" (figure~\ref{fig:CountdownCode}) was written to test out a number of parts within the emulator, such as comparisons and loops, with a semi-realistic function. The program takes an input and counts down until 0 printing each number as it goes. There may still be errors in the emulation in regards to more complex instructions using post/pre increments/decrements and indirection in unusual ways, however I feel all those mistakes have been found from the testing we've conducted.

\begin{figure}[h]
\begin{code}
      CALL Main
      HALT
Main: CALL read
   C: CALL print
      BGT S
      RETURN	
   S: SUB #1 A
      JUMP C
\end{code}
    \caption{Code for countdown.ass program.}
    \label{fig:CountdownCode}
\end{figure}

\subsection{GUI}

Figures~\ref{fig:screen_shot} and~ \ref{fig:screen_shot_running} shows the final product when the application is in edit mode and when it is running the emulation.

\begin{figure}[h!]
    \centering
    \includegraphics[width=300px]{screen_shot}
    \caption{Screenshot of the Final Product}
    \label{fig:screen_shot}
\end{figure}

\begin{figure}[h!]
    \centering
    \includegraphics[width=300px]{screen_shot_running}
    \caption{Screenshot of the Final Product Running The Emulation}
    \label{fig:screen_shot_running}
\end{figure}

\newpage

\section{Conclusion}

\subsection{Callum McColl's Reflection}

During the creation of the GUI I felt that haskell allowed me to accomplish much more in much less code.  It also allowed me to get to grips with the differences in how one might go about an implementation in a procedural language versus a functional language.  For instance some tasks which I found to be trivial in a procedural language became quite cumbersome for the uninitiated in the functional world.  One example of this is with the counter that is used to keep track of which step the user is on in the emulation.  In a procedural language you would simply create an int and increment it and that would be that, however in a functional world, things get monaddy really quick.

One of the things that I will take from this project is that it further progressed my knowledge with respect to Functors, Applicatives and Monads.  At the start of the semester I still did not have a firm grasp on these concepts but by the end, I think that I have a firm understanding.

The main chore of haskell (that I found) is the fact that I must pass the things I need to every function I call, and since I have a lot of functions, it means that I am passing around the same stuff all the time.  In the Main module I am passing around: the assembly source code, the emulation counter, the array of environments, a bool indicating whether the app is running and the container that I am drawing everything in.  Now it became obvious to me after I completed the implementation that I should have created a separate data structure which contained these things and simply passed that around, but then I would have to deal with the fact that I cannot mutate this thing and if I want to change anything, I would have to create a new one.  However, I do think that this is better than the current approach.

The type system caused some grief for me as it can be strict when I don't want it to be and very loose when I want it to be strict.  An example of this is in the Icons module.  In the Icons module I have defined several icons which have a type of \highlighttt{StockId}.  \highlighttt{StockId} itself is just a type synonym for \highlighttt{String}.  But does this mean that I can create a \highlighttt{StockId} from a string literal?

\begin{examplecode}
    playButton :: StockId
    playButton = "media-playback-start"
\end{examplecode}

\noindent Nope:

\begin{examplecode}
    playButton :: StockId
    playButton = fromString "media-playback-start"
\end{examplecode}

\noindent These sorts of problems become hard to solve for someone who may not be very familiar with the haskell Prelude, but this is more of an experience with the language thing as opposed to a gripe about functional programming.

What I loved about functional programming is the higher order functions.  Just with the use of \highlighttt{map}, you can hide away the fact that you have many things or just one.  Many times the answer to a problem was "Oh, I'll just map this function over it".  The higher order functions are very powerful and very easy to use.

Overall the functional programming concepts that I learned are generally orthogonal to the concepts of object oriented programming and as such can be used in traditional programming languages.  Therefore I see the concepts taught and learned in the course as extra tools to use as a software developer.

\subsection{Andrew Paroz's Reflection}
On a whole I feel this project ran fairly well. I know out of everyone in the group I had the most free time since this was basically my only course, so I tried to jump in an do something wherever I could. Between Callum and myself things worked well and we were able to put our two parts together at the end fairly easily. We've worked together many times before so this was really nothing new to us. Using Slack to communicate between everyone seemed to be really useful and I found it a really good way to get in contact with others right away when I was having problems.

This course was the first time I'd learnt about and used a Functional programming language like Haskell. By the end of the course I'd like to say that I am at least competent at writing some basic Haskell, though it's pretty clear there is still lots to learn about it. As a programming language I'm not sure what to think, as normally I program in C++. It is clearly a good language for isolating errors and preventing the sort of side-effect mistakes that happen frequently in a large C++ program. It is also really good for breaking down a problem, as you can write and test smaller functions and build upwards instead of downwards. Assuming you can get your environment and libraries set up (which I never found too hard), it is also great for multi-platform programming, as  everything within Haskell is the same across every platform.

The only big problem I found with Haskell was that it is not good for things like user interaction and interfaces. I found that as soon as you start working with an  interface, you are back at side effects and you might as well not be using a functional language. Of course I had problems learning the syntax and the style in which you write things, same with any new language, but once I started getting the hang of it, everything tended to go well. Not sure if I will be using Haskell in the future, but certainly something I'll keep in mind when planning for new projects.

\end{document}



